%%%%%Präambel%%%%%

\documentclass[12pt,a4paper]{article}%Schriftgröße, Papierformat einstellen
%\documentclass{scrbook}
\usepackage[top=30mm,bottom=30mm]{geometry}
\usepackage{lipsum}
%Pakete laden zur deutschen Rechtschreibung und für Umlaute
\usepackage[T1]{fontenc}
\usepackage[ngerman]{babel}
\usepackage[utf8]{inputenc} %für Windows, Linux
%\usepackage[applemac]{inputenc} %für Mac
%\usepackage{xcolor}
\usepackage[dvipsnames]{xcolor}
\usepackage{cancel}
\usepackage{titlesec}
\usepackage{cite}
\usepackage{filecontents}
\usepackage{tabularx}
\usepackage{harvard}
\usepackage{units}
\usepackage{longtable} 
\usepackage{chngcntr}
\usepackage{stmaryrd}
\usepackage{array}
\let\harvardleftorig\harvardleft
%\usepackage[round]{natbib}
%\usepackage{hyperref}
\usepackage[nottoc,numbib]{tocbibind}
\usepackage{siunitx}
\usepackage{esvect}

%Pakete laden zu mathematischen Symbolen etc.
\usepackage{calc} 
\usepackage{amsmath,amssymb,amsthm}
\usepackage{scrpage2}
\pagestyle{scrheadings}
\clearscrheadfoot
\automark[chapter]{section}
\ofoot{\pagemark}
\ifoot{Florian Leuze}
\chead{\headmark}
\setfootsepline{1pt}
\setheadsepline{1pt}
%\setheadsepline[\textwidth+20pt]{0.5pt}

%Inhaltsverzeichnis mit Links erstellen
\usepackage[colorlinks,
pdfpagelabels,
pdfstartview = FitH,
bookmarksopen = true,
bookmarksnumbered = true,
linkcolor = black,
plainpages = false,
hypertexnames = false,
citecolor = black] {hyperref}

% Umgebungen für Definitionen, Sätze, usw.
\newtheorem{satz}{Satz}[section]
\newtheorem{definition}[satz]{Definition}     
\newtheorem{lemma}[satz]{Lemma}	
% Es werden Sätze, Definitionen etc innerhalb einer Section mit
% 1.1, 1.2 etc durchnummeriert, ebenso die Gleichungen mit (1.1), (1.2) ..                  
\numberwithin{equation}{section}

\setcounter{secnumdepth}{4}
\setcounter{tocdepth}{4}

\titleformat{\paragraph}
{\normalfont\normalsize\bfseries}{\theparagraph}{1em}{}
\titlespacing*{\paragraph}
{0pt}{3.25ex plus 1ex minus .2ex}{1.5ex plus .2ex}

\DeclareMathOperator{\grad}{grad}
\DeclareMathOperator{\diverg}{div}
\DeclareMathOperator{\rot}{rot}
\DeclareMathOperator{\spur}{spur}
\DeclareMathOperator{\determ}{det}

%neue Befehle definieren
\newcommand{\R}{\mathbb{R}} %zB \R als Abkürzung für das Symbol der reellen Zahlen
\newcommand{\N}{\mathbb{N}}
\newcommand{\Z}{\mathbb{Z}}
\newcommand{\Q}{\mathbb{Q}}
\newcommand{\C}{\mathbb{C}}
\newcommand{\diffp}{\partial}
\newcommand{\laplace}{\Delta}

\newcommand{\subsubsubsection}{\paragraph}
\newcommand\citevgl
{\def\harvardleft{(vgl.\ \global\let\harvardleft\harvardleftorig}%
 \cite
}
\newcommand\citeVgl
{\def\harvardleft{(Vgl.\ \global\let\harvardleft\harvardleftorig}%
 \cite
}

\newcolumntype{L}[1]{>{\raggedleft\let\newline\\\arraybackslash\hspace{0pt}}m{#1}}
%Makros
%Makro Color
%#1 Text
\def\colBord#1{\begingroup\color{Fuchsia}{#1}\endgroup}
\def\colRed#1{\begingroup\color{Red}{#1}\endgroup}
\def\colGreen#1{\begingroup\color{LimeGreen}{#1}\endgroup}
\def\colBlue#1{\begingroup\color{NavyBlue}{#1}\endgroup}

\def\usGreen#1#2{\underset{\colGreen{#1}}{#2}}
\def\usBord#1#2{\underset{\colBord{#1}}{#2}}

\def\ubGreen#1#2{\underbrace{#2}_{\colGreen{#1}}}

\def\defF{\textbf{Def.: }}

\def\vecT#1{\left(\begin{array}{c} #1 \end{array}\right)}
\def\dddot{\cdot \\ \cdot \\ \cdot}
\def\vecD#1{\vecT{#1_1 \\ \dddot \\ #1_d}}
\def\epsF{\pmb{\varepsilon}}

\def\multiTwo#1#2{\multicolumn{2}{>{\hsize=\dimexpr2\hsize+2\tabcolsep+\arrayrulewidth\relax}#1}{#2}}
\def\multiThree#1#2{\multicolumn{3}{>{\hsize=\dimexpr3\hsize+4\tabcolsep+2\arrayrulewidth\relax}#1}{#2}}

\def\inR#1{\qquad ,\; #1 \in \R}
\def\bracks#1{\left[ #1 \right]}
\def\abs#1{\left| #1 \right|}
\def\brac#1{\left( #1 \right)}

%laziness
\def\fermi{Fermi-Dirac-Verteilung}

\newcolumntype{L}[1]{>{\raggedleft\let\newline\\\arraybackslash\hspace{0pt}}m{#1}}
\newcolumntype{R}[1]{>{\raggedright\let\newline\\\arraybackslash\hspace{0pt}}m{#1}}



\def\formTab#1#2{
\begin{equation}
  \begin{tabularx}{12cm}{R{3cm} l l}
    #1 &: &$#2$
  \end{tabularx}
\end{equation}
}
\newcommand{\formTabL}[3]{
\begin{equation}
  \begin{tabularx}{12cm}{R{3cm} l l}
    #1 &: &$#2$ 
  \end{tabularx}
  \label{eq:#3}
\end{equation}}
\def\formTn{$ \\ $\;$ & $\;$ & $}
\def\formTnQ{$ \\ $\;$ & $\;$ & $\qquad}
\def\formTnQQ{$ \\ $\;$ & $\;$ & $\qquad \qquad}
\def\formTnQQQ{$ \\ $\;$ & $\;$ & $\qquad \qquad \qquad}

\renewcommand{\theequation}{\arabic{section}.\arabic{subsection}
.\arabic{equation}}
%Setzt den equation-Zaehler nach jeder Seite zurueck
\numberwithin{equation}{subsection}	


%\setlength\abovedisplayskip{0pt}

% Auf der Seite http://detexify.kirelabs.org/classify.html können Sie mathematische Symbole, Pfeile usw per Maus eingeben und bekommen den Latex-Befehl dafür angezeigt.
% detexify gibt es auch als App...

%jetzt beginnt das eigentliche Dokument
\begin{document}

\bibliographystyle{agsm}

\author{}
\title{\underline{Mikroelektronik II Formelsammlung} \\ $\;$ \\ $\;$ \\ Florian Leuze}
\date{}

\maketitle % erzeugt den Kopf
\newpage
\tableofcontents

  \section*{Versionierung}
  \begin{tabular}{|p{2cm}|p{1cm}|p{1.5cm}|p{8.5cm}|}\hline
    Datum & Vers. & Kürzel & Änderung \\ \hline
    28.08.2018 & 0.1 & FL & Erzeugung Dokument; Erzeugung Inhaltsverzeichnis; Erzeugung Versionierung; Erzeugung Allg., PN Üb., Bohrsch., Bipol., MOSFET\\ \hline
    29.08.2018 & 0.3 & FL & Korrektur Ionisationsenergie(d,e); erzeugt Inversionsspannung(a,b)\\ \hline 
  \end{tabular}
  
  
\newpage
\section{Mikroelektronik II}
\subsection{Allgemeines}
  \subsubsection{Spezifischer Widerstand}
  \formTabL{Ohmsches Gesetz}{V = RI \bracks{V}}{ohms_law_V}
  \formTabL{Widerstand}{R = \frac{ V}{I} \bracks{\Omega}}{ohms_law_R}
  \formTabL{Leitwert}{G = \frac{1}{G} = \frac{I}{V} \bracks{\frac{1}{\Omega} = S}}{leitwert}
  \formTabL{Spezifischer Widerstand}{R = \rho \frac{L}{A} \bracks{\Omega}}{spezR}
  \formTabL{Leitfähigkeit(a)}{\sigma \frac{A}{L}\bracks{S}}{leitf_a}
  \formTabL{Leitfähigkeit(b)}{\sigma = \frac{1}{\rho}}{leitf_b}
  \subsubsection{Elektrostatik}
  \formTabL{Poisson Gleichung(a)}{\laplace \varphi (\vv{r}) = \frac{\varrho_Q (\vv{r})}{\varepsilon}}{poisson_a}
  \formTabL{Poisson Gleichung(b)}{\nabla \vv{\varepsilon} (\vv{r}) = -\displaystyle\frac{\varrho_Q(\vv{r}}{\varepsilon}}{poisson_b}
  \formTabL{El. Feld}{\vv{\varepsilon}(\vv{r}) = - \grad \varphi(\vv{r}) = -\vv{\nabla}(\vv{r})}{elFeld_a}
  \formTabL{Coulomb Kraft}{\vv{F_c}(\vv{r}) = \displaystyle\frac{1}{4 \pi \varepsilon_0} \frac{q^2}{\sqrt{x^2 + y^2 + z^2}} \vecT{x \\ y \\ z}}{coulomb}
  
\subsection{PN Übergang}
\subsubsection{Quasi-Fermi-Niveaus (QFN oder Imref)}
  \formTab{Elektronenkonzentration}{n = N_C e^{-\displaystyle \frac{E_C - E_{Fn}}{kT}}}
  \formTab{Lochkonzentration}{p = N_V e^{-\displaystyle \frac{E_{Fp} - E_V}{kT}}}
  \formTabL{Massenwirkungsgesetz(b)}{np = N_C N_V e^{\frac{E_g}{kT}} e^{\frac{E_{Fn} - E_{Fp}}{kT}} \formTnQQQ = n_i^2 e^{\displaystyle \frac{E_{Fn} - E_{Fp}}{kT}}}{massenwirkung_b_qfn}
  \subsubsection{Diffusionsspannung und Weite der RLZ}
  \formTabL{Flächenladung gesamt}{G_{F,total} = Q_L + Q_R = 0 \formTnQQQ = -qN_A d_p + q N_D d_n}{gf_total}
  \vspace{-0.2cm}
  \formTabL{Folgerung aus \eqref{eq:gf_total}(a)}{N_A d_p = N_D d_n}{flaechenladung_folg_1}
  \formTab{Folgerung aus \eqref{eq:gf_total}(b)}{\displaystyle\frac{d_p}{d_n} = \frac{N_D}{N_A}}
  \formTabL{Weite(a)}{w = d_n + d_p}{w1}
  \formTab{Diffusionsspannung(a)}{V_{bi} = \displaystyle\varphi_R(d_n) = 	-\frac{\epsF_{max}}{2}(d_n + d_p) = -\frac{\epsF_{max}}{2} w}
  Mit \eqref{eq:eFeld_max} erhält man
  \formTab{Diffusionsspannung(b)}{V_{bi} = \displaystyle\frac{qN_A}{2\varepsilon_d}d_p(d_n + d_p)}
  
  Mit \eqref{eq:flaechenladung_folg_1} erhält man schließlich 
  \formTabL{Teilweite $d_n$}{d_n = \displaystyle\sqrt{\frac{2\varepsilon_d N_A V_{bi}}{qN_D (N_A + N_D)}} \sim \sqrt{V_{bi}}}{teilweite_a}  
  \formTabL{Teilweite $d_p$}{d_p = \displaystyle\sqrt{\frac{2\varepsilon_d N_D V_{bi}}{qN_A (N_A + N_D)}} \sim \sqrt{V_{bi}}}{teilweite_b}  
  
  Bei angelegter ext. Spannung gilt bei \eqref{eq:teilweite_a} und \eqref{eq:teilweite_b} $V_{bi,neu} = V_{bi} + V_{DS}$.
  
  \formTab{Weite(b)}{w = \displaystyle dp\left(1+\frac{N_A}{N_D}\right)}
  \formTabL{Weite(c)}{\displaystyle w = \sqrt{\frac{2\varepsilon_d}{q} \frac{N_A + N_D}{N_A N_D}V_{bi}} \sim \sqrt{V_{bi}}}{rlz_w_c}
  \formTab{Weite Schottky}{w = x_{n,p} = \sqrt{\displaystyle\frac{2 \varepsilon_d |\Phi_{MS}|}{q N_D}}}
  \formTab{Folgerung(a)}{w \uparrow \quad \Rightarrow \quad V_{bi} \uparrow}
  
  \formTab{Diffusionsspannung(c)}{\displaystyle qV_{bi} = kT\;ln\left(\frac{n_{n0}p_{p0}}{n_i^2}\right) \approx kT\;ln\left(\frac{N_A N_D}{n_i^2}\right)}
  \formTab{Diffusionsspannung(d)}{\displaystyle V_{bi}^0 = \frac{kT}{q}ln\left( \frac{N_D N_A}{n_i^2}\right)}
  \formTabL{Diffusionsspannung(e)}{V_{bi} = V_{bi}^0 - V}{diffspann_e}
  \formTabL{Diffusionsspannung(f)}{V_{bi} = \displaystyle\frac{q}{2\varepsilon_d}(N_A x_p^2 + N_D x_n^2)}
  Aus \eqref{eq:flaechenladung_folg_1} und \eqref{eq:w1} folgen
  \formTab{Relative Weite $p$}{d_p = w \frac{N_D}{N_A + N_D}}
  \formTab{Relative Weite $n$}{d_n = w \frac{N_A}{N_A + N_D}}
  \formTab{Flächenladungsdichte}{Q_F = \abs{Q_R} = \abs{Q_L} = qN_A d_p = qN_D d_n \formTnQQQ = q\frac{N_A N_D}{N_A + N_D}w = \sqrt{2\varepsilon_d \frac{N_A N_D}{N_A + N_D}q V_{bi}}}
  
  \subsubsection{Feldstärke}
  \formTabL{Maximale el. Feldstärke(a)}{\epsF_{max} = \epsF (x = 0) = \epsF_{L,R} (x = 0)}{eFeld_max}
  \formTab{Maximale el. Feldstärke(b)}{\left|\epsF_{max}\right| = 2\frac{V_{bi}}{w}= \frac{qN_{D,A}}{\varepsilon_d}x_{n,p}}
  \subsubsection{Massenwirkungsgesetz}
  \formTabL{Massenwirkungsgesetz(a)}{np = N_C N_V e^{-\frac{E_C-E_F}{kT}} e^{-\frac{E_F-E_V}{kT}}}{massenwirkung_a}
  \formTabL{Bandlücke}{E_g = E_C-E_V}{e_gap}
  \eqref{eq:e_gap} in \eqref{eq:massenwirkung_a} (wobei $n_i$ die intrinsische Ladungsträgerdichte ist):
  \formTabL{Massenwirkungsgesetz(b)}{np = N_C N_V e^{\frac{E_g}{kT}} = n_i^2}{massenwirkung_b}
  Also gilt ganz allgemein (und zwar unabhängig von der Dotierung)
  \formTabL{Massenwirkungsgesetz(c)}{np = n_i^2}{massenwirkung_c}
  \formTab{Intrinsische Ladungsträgerdichte}{ni = \sqrt{N_C N_V} \cdot e^{-\frac{E_g}{2kT}}}
  
  \subsubsubsection{Umformulierungen}
  \formTab{$E_C - E_F$}{E_C - E_F = kT \cdot ln\left(\frac{N_C}{n}\right)}
  \formTab{$E_F - E_V$}{E_F - E_V = kT \cdot ln\left(\frac{N_V n}{n_i^2}\right)}
  
  \subsubsection{Energiebetrachtung}
  \formTab{Austrittsarbeit a)}{\Phi_{MS} = \Phi_M - \Phi_S}
  \formTab{Austrittsarbeit b)}{V_{bi} = \Phi_{MS}}
  \newpage
  \subsection{Bohrsches Atommodell}
  \subsubsection{Energie}
  \formTab{Energie i-te Schale}{E_i = - \frac{1}{8 \pi \varepsilon_0} \frac{q^2}{r_i}}
  \formTab{Energie Schalenübergang(a)}{E_{ae} = E_a - E_e = \displaystyle\frac{q^2}{8 \pi \varepsilon_0}\left(\frac{1}{r_e} - \frac{1}{r_a}\right)}
  \formTab{Bahnradius(a), Bohrscher Radius}{r_1 = \displaystyle\frac{4 \pi \varepsilon_0 \hbar^2}{q^2m}}
  \formTab{Bahnradius(b)}{r_i = n^2 \cdot r_1}
  \formTab{Energie Schalenübergang(b)}{E_{ae} = \displaystyle\frac{q^2}{8 \pi \varepsilon_0} \frac{q^2 m}{4 \pi \varepsilon_0 \hbar^2} \left(\frac{1}{n_e^2} - \frac{1}{n_a^2}\right)}
  \formTab{Frequenz Schalenübergang(a)}{f_{ae} = \displaystyle\frac{mg^4}{(4 \pi \hbar)^3\varepsilon_0^2} \left(\frac{1}{n_e^2} - \frac{1}{n_a^2}\right)}
  \formTab{Rydberg-Ritz Formel(a)}{\frac{1}{\lambda_{ae}} =  \displaystyle\frac{mg^4}{(4\pi \hbar)^3\varepsilon_0^2 C_0} \left(\frac{1}{n_e^2} - \frac{1}{n_a^2}\right)}
  \formTab{Rydberg Konstante}{R_\infty = \displaystyle\frac{mg^4}{(4\pi \hbar)^3\varepsilon_0^2 C_0} = \frac{mq^4}{8h^3 \varepsilon_0^2 C_0}}
  \formTab{Wellenlänge Schalenübergang(a)}{\lambda_{ae} = \left( R_\infty \left( \frac{1}{n_e} - \frac{1}{n_a}\right) \right)^{-1}}
  Für wasserstoffähnliche Atome wird die Rydberg Ritz Formel angepasst:
  \formTab{Rydberg-Ritz Formel(b)}{\displaystyle\frac{1}{\lambda_{ae}} =  \displaystyle R_\infty Z^2 \left(\frac{1}{n_e^2} - \frac{1}{n_a^2}\right)}
  wobei für $Z$ die Kernladungszahl einzusetzen ist, was bei wasserstoffähnlichen Atomen die Ordnungszahl ist.
  \formTab{Bahngeschwindigkeit}{v_i = \frac{\hbar}{m r_i}}
  \formTab{Ionisationsenergie(a)}{E_{n\infty} = -R_\infty h C_0 \frac{1}{n^2}}
  \formTab{Ionisationsenergie(b)}{E_{1\infty} = -R_\infty h C_0}
  \formTab{Ionisationsenergie(c)}{E_{1\infty} = -R_\infty h C_0 Z^2}
  \formTab{Schale}{n = \sqrt{-\displaystyle\frac{R_\infty h C_0}{E_{n\infty}}}}
  \formTab{Kernladungszahl}{Z = \sqrt{\displaystyle\frac{1}{\lambda_{ae}R_\infty}\left( \frac{1}{n_e^2}-\frac{1}{n_a^2}\right)^{-1}}}
  Betrachtet man ein in ein anderes Element eingebrachtes Elektron müssen die Zusammenhänge angepasst werden.
  \formTab{Bahnradius(c)}{r_1 = \displaystyle\frac{4 \pi \varepsilon_0 \varepsilon_{rel}\hbar^3}{q^2 m^*}}
  \formTab{Ionisationsenergie(d)}{E_{ae} = R_\infty h C_0 \displaystyle\frac{m^*}{m_0\cdot \varepsilon_{rel}^2} \left(\frac{1}{n_e^2} - \frac{1}{n_a^2}\right)}
  \formTab{Ionisationsenergie(e)}{E_{1\infty} = -R_\infty h C_0 \displaystyle\frac{m^*}{m_0\cdot \varepsilon_{rel}^2}}
  \subsection{Bipolartransistor}
  \subsubsection{Leistung}
  \formTab{Leistung(a)}{P_{BT} \cong \frac{N}{2} I V_{cc} + N \frac{I}{\beta} V_{cc}}
  \formTab{Leistung(b)}{P_{BT} \cong \left(\frac{1}{2}+\frac{1}{\beta}\right) N I V_{cc} = \frac{2 + \beta}{2 \beta}N I V_{cc}}
  \subsubsection{Emitterwirksamkeit/Transportfaktor}
  \formTab{Emitterwirksamkeit(a)}{\alpha_E = \frac{I_n}{I_n + I_p}}
  \formTab{Emitterwirksamkeit(b)}{\alpha_e = \displaystyle\frac{1}{1+\frac{L_n D_p N_A^{(B()}}{L_p D_n N_D^{(E)}}}}
  \formTab{Emitterstrom}{|I_E| = A_q D_b \displaystyle\frac{n(0)-0}{w_B}}
  \formTab{Basisstrom}{|I_B| = Aq \displaystyle\frac{n(0)w_b}{2}\frac{1}{\tau_n}}
  \formTab{Transportfaktor}{\alpha_T = \displaystyle\frac{|I_C|}{|I_B|} = 1 - \frac{1}{2} \left(\frac{w_B}{L_n}\right)^2}
  \formTabL{Diffusionslänge Löcher}{L_p = \sqrt{D_p \tau_p}}{difflang_p} 
  \formTabL{Diffusionslänge Elektronen}{L_n = \sqrt{D_n \tau_n}}{difflang_n}
  \subsubsection{Verstärkung}
  \formTab{Stromverstärkung}{\beta_0 = \frac{I_C}{I_B}}
  \formTab{Stromverstärkung Emitterschaltung}{\beta_0 = \displaystyle\frac{I_C}{I_E - I_C} = \frac{\alpha_0}{1-\alpha_0} = \displaystyle\frac{\alpha_E \alpha_T}{1 - \alpha_E \alpha_T}}
  
  \subsection{MOSFET}
  \subsubsection{Leistung}
  \formTab{Leistung(a)}{P_{NMOS} \cong \frac{N}{2}I V_{cc} + N I_{dyn} V_{cc}}
  \formTab{Leistung(b)}{P_{CMOS,ideal} \cong N C_L f V_{cc}^2}
  Hierbei gilt $I_{dyn}$ (MOSFET) $ < \frac{I}{\beta}$ (BPT).
  \formTab{Dyn. Strom}{I_{dyn} = f C_L V_{cc}}
  \formTab{Power-Delay-Produkt}{\displaystyle\frac{P_{CMOS,ideal,max}}{C_l V_{cc}^2} = N f_{max} = konst.}
  \subsubsection{Gate Kapazität}
  \formTab{Umladegeschwindigkeit}{v_D = \mu \varepsilon = \mu \displaystyle\frac{V_{DS}}{L}}
  \formTab{Steilheit}{g_m = \frac{W C_{gox\mu}}{L}}
  \formTab{Umladedauer(a)}{\tau_L = \displaystyle\frac{L^2}{\mu V_{DS}}}
  \formTab{Umladedauer(b)}{\tau_L = \displaystyle\frac{W C_{gox}L}{g_m}}
  \formTab{Inversionsspannung(a)}{V_{sl} = \frac{d_I}{\varepsilon}\left( 2 q \left( \sqrt[3]{N_A}\right)^2+\sqrt{qN_A \varepsilon \beta ln\left(\frac{N_A}{n_i}\right)}\right)\formTnQQQ +\beta ln\left(\frac{N_A}{N_i}\right)}
  \formTab{Inversionsspannung(b)}{V_{Sl} = V_{Isolator} + V_{T,ideal} = V_{Isolator} + 2\Psi_B}
  \subsubsection{Takt}
  \formTab{Taktfrequenz}{f_T = \displaystyle\frac{1}{2 \pi \tau_L} = \frac{1}{2\pi} \frac{g_m}{W C_{gox}L}}
  \newpage
\section{\underline{Anhänge}}
	\subsection{Abkürzungen/Formelzeichen} \label{ch:names}
	\renewcommand{\arraystretch}{1.5}
	\begin{longtable} {|p{2cm}|p{3cm}|p{8.4cm}|} \hline
	% Definition des Tabellenkopfes auf der ersten Seite
	%Spaltenbezeichnungen
	\textbf{Zeichen} & \textbf{Einheit} & \textbf{Bedeutung} \\
	\hline
	\endfirsthead % Erster Kopf zu Ende
	% Definition des Tabellenkopfes auf den folgenden Seiten
	\caption{Abkürzungen/Formelzeichen}\\ \hline
	%Spaltenbezeichnungen
	\textbf{Zeichen} & \textbf{Einheit} & \textbf{Bedeutung} \\
	\hline
	\endhead % Zweiter Kopf ist zu Ende
	\multicolumn{3}{r}{Fortsetzung auf Folgeseite}\\
	\endfoot
	\hline
	%\multicolumn{3}{r}{Ende} \\
	\endlastfoot
	
	%a-g
	$A$ & $m^2$ & Fläche \\ \hline
	$a$ & $\frac{m}{s^2}$ & Beschleunigung \\ \hline
	$b$ & $\frac{cm^2}{Vs}$ & Ladungsträgerbeweglichkeit \\ \hline
	$d$ & $m$ & Dicke \\ \hline
	$D_n$ & $\frac{m^2}{s}$ & Diffusionskonstante für Elektronen \\ \hline
	$D_p$ & $\frac{m^2}{s}$ & Diffusionskonstante für Löcher \\ \hline
	$e$ & $C$ & Elementarladung \\ \hline
	$E$ & $\frac{N}{C} = \frac{VAs}{mAs} = \frac{V}{m}$ & Elektrische Feldstärke \\ \hline
	$E_c$ & $eV$ & Leitungsbandkante \\ \hline
	$E_F$ & $eV$ & Fermi-Energie \\ \hline
	$E_g$ & $eV$ & Energie der Bandlücke \\ \hline
	$E_v$ & $eV$ & Valenzbandkante \\ \hline
	$f$ & $Hz$ & Frequenz \\ \hline
	$\vec{F}$ & $N = \frac{kgm}{s^2}$ & Kraft \\ \hline
	$G$ & $\frac{A}{V} = \frac{1}{\Omega} = S$ & Leitwert \\ \hline
	
	%h-n
	$h$ & $eVs$ & Plank-Konstante\\ \hline
	$\hbar$ & $eVs$ & Planksches Wirkungsquantum \\ \hline
	$i$ & $A$ & Elektrischer Strom \\ \hline
	$j$ & $\frac{A}{m2}$ & Elektrische Stromdichte \\ \hline
	$J_n$ & $\frac{A}{m2}$ & Elektronenstromdichte \\ \hline
	$J_p$ & $\frac{A}{m2}$ & Löcherstromdichte \\ \hline
	$J_{diff}$ & $\frac{A}{m2}$ & Diffusionsstromdichte \\ \hline
	$J_{part}$ & $\frac{A}{m2}$ & Partikelstromdichte \\ \hline
	$J_to$ & $\frac{A}{m2}$ & Totale Stromdichte \\ \hline
	$J_r$ & $\frac{A}{m2}$ & Rekombinationsstromdichte \\ \hline
	$J_{drift}$ & $\frac{A}{m2}$ & Driftstromdichte \\ \hline
	$l$ & $m$ & Länge \\ \hline
	$L$ & $m$ & Minoritätsladungsträgerdiffusionslänge \\ \hline
	$L_n$ & $m$ & Diffusionslänge Elektronen \\ \hline
	$L_p$ & $m$ & Diffusionslänge Löcher \\ \hline
	
	%m-u
	$n$ & ... & Elektronenkonzentration \\ \hline
	$n_i$ & ... & Intrinsische Ladungsträgerdichte \\ \hline
	$n_{id}$ & ... & Idealität einer Diode \\ \hline
  $N_A$ & $m^{-3}$ & Akzeptorendichte \\ \hline
  $N_D$ & $m^{-3}$ & Donatorendichte \\ \hline
	$N_C$ & $cm^{-3}$ & Effektive Zustandsdichte der Elektronen \\ \hline
	$N_V$ & $cm^{-3}$ & Effektive Zustandsdichte der Löcher \\ \hline
	$p$ & ... & Lochkonzentration \\ \hline
	$q$ & $C$ & Probeladung (in der Regel = $e$) \\ \hline
	$\vec{r}$ & $m$ & Weg \\ \hline
	$r$ & $\Omega$ & Differentieller Widerstand \\ \hline
	$R$ & $\Omega$ & Widerstand \\ \hline
	$R_F$ & $\frac{\Omega}{square}$ & Flächenwiderstand \\ \hline 
	$U$ & $V$ & Elektrische Spannung \\ \hline
	$U_g$ & $V$ & Gesamtspannung \\ \hline
	 
	%v-z
	$v$ & $\frac{m}{s}$ & Geschwindigkeit \\ \hline
	$v_D, v_d$ & $\frac{m}{s}$ & Driftgeschwindigkeit \\ \hline
	$w$ & $m$ & Weite bzw. Breite  \\ \hline
	$W$ & $Ws = J = \frac{kgm^2}{s^2}$ & Arbeit bzw. Energie \\ \hline
	
	%griechisch
	$\alpha$ & $\frac{1}{^{\circ} C}$ & Temperturkoeffizient des Ohmwiderstandes \\ \hline
	$\nu$ & $Hz$ & Hier Frequenz der Welle \\ \hline
	$\rho$ & $\frac{V cm}{A} = \Omega  cm$ & Spezifischer Widerstand \\ \hline
	$\rho_e$ & ... & Ladungsdichte \\ \hline
	$\kappa$ & $\frac{1}{\Omega cm} = \frac{S}{cm}$ & Spezifische Leitfähigkeit \\ \hline
	$\varepsilon_0$ & $\frac{As}{Vm}$ & Dielektrizitätskonstante im Vakuum \\ \hline
	$\varphi$ & $V$ & Elektrisches Potential \\ \hline
	$\tau$ & $s$ & Stoßzeit \\ \hline
	$\tau$ & $s$ & Minoritätsladungsträgerlebensdauer \\ \hline
	$\mu$ & $\frac{cm^2}{Vs}$ & Beweglichkeit \\ \hline
	%Sonderzeichen
	\end{longtable}
	\renewcommand{\arraystretch}{1}
	
	\subsection{Wichtige Donatoren und Akzeptoren} \label{ch:don/acc}
	\renewcommand{\arraystretch}{1.5}
	\begin{longtable} {|p{2cm}|p{3cm}|p{8.4cm}|} \hline
	% Definition des Tabellenkopfes auf der ersten Seite
	%Spaltenbezeichnungen
	\textbf{Ch. Sym.} & \textbf{Name} & \textbf{Typ} \\
	\hline
	\endfirsthead % Erster Kopf zu Ende
	% Definition des Tabellenkopfes auf den folgenden Seiten
	\caption{Wichtige Donatoren und Akzeptoren}\\ \hline
	%Spaltenbezeichnungen
	\textbf{Zeichen} & \textbf{Einheit} & \textbf{Bedeutung} \\
	\hline
	\endhead % Zweiter Kopf ist zu Ende
	\multicolumn{3}{r}{Fortsetzung auf Folgeseite}\\
	\endfoot
	\hline
	%\multicolumn{3}{r}{Ende} \\
	\endlastfoot
	$B$ & Bor & Akzeptor \\ \hline
	$Al$ & Alluminium & Akzeptor \\ \hline
	$Ga$ & Gallium & Akzeptor \\ \hline
	$In$ & Indium & Akzeptor \\ \hline
	$P$ & Phosphor & Donator \\ \hline
	$As$ & Arsen & Donator \\ \hline
	$Sb$ & Antimon & Donator \\ \hline
	$Bi$ & Wismut & Donator \\ \hline
	\end{longtable}
	\renewcommand{\arraystretch}{1}
	\newpage
	\subsection{Effektive Massen} \label{ch:don/acc}
	\renewcommand{\arraystretch}{1.5}
	\begin{longtable} {|p{2cm}|p{3cm}|p{8.4cm}|} \hline
	% Definition des Tabellenkopfes auf der ersten Seite
	%Spaltenbezeichnungen
	\textbf{Band} & \textbf{Wert} & \textbf{Element} \\
	\hline
	\endfirsthead % Erster Kopf zu Ende
	% Definition des Tabellenkopfes auf den folgenden Seiten
	\caption{Effektive Massen}\\ \hline
	%Spaltenbezeichnungen
	\textbf{Band} & \textbf{Wert} & \textbf{Element} \\
	\hline
	\endhead % Zweiter Kopf ist zu Ende
	\multicolumn{3}{r}{Fortsetzung auf Folgeseite}\\
	\endfoot
	\hline
	%\multicolumn{3}{r}{Ende} \\
	\endlastfoot
	$\frac{m_n^*}{m_0}$ & $1,08$ & Silizium \\ \hline
	$\frac{m_n^*}{m_0}$ & $1,561$ & Germanium \\ \hline
	$\frac{m_n^*}{m_0}$ & $1,067$ & Gallium-Arsenid \\ \hline
	$\frac{m_p^*}{m_0}$ & $1,10$ & Silizium \\ \hline
	$\frac{m_p^*}{m_0}$ & $1,291$ & Germanium \\ \hline
	$\frac{m_p^*}{m_0}$ & $1,473$ & Gallium \\ \hline
	\end{longtable}
	\renewcommand{\arraystretch}{1}
	
	\subsection{Bandlücken wichtiger Materialien} \label{ch:gaps}
	\renewcommand{\arraystretch}{1.5}
	\begin{longtable} {|p{2cm}|p{3cm}|p{8.4cm}|} \hline
	% Definition des Tabellenkopfes auf der ersten Seite
	%Spaltenbezeichnungen
	\textbf{Zeichen} & \textbf{Wert in \text{e}V} & \textbf{Material} \\
	\hline
	\endfirsthead % Erster Kopf zu Ende
	% Definition des Tabellenkopfes auf den folgenden Seiten
	\caption{Bandlücken wichtiger Materialien}\\ \hline
	%Spaltenbezeichnungen
	\textbf{Zeichen} & \textbf{Einheit} & \textbf{Bedeutung} \\
	\hline
	\endhead % Zweiter Kopf ist zu Ende
	\multicolumn{3}{r}{Fortsetzung auf Folgeseite}\\
	\endfoot
	\hline
	%\multicolumn{3}{r}{Ende} \\
	\endlastfoot
	$E_{g,SiO_2}$ & $9$ & Siliziumdioxid \\ \hline
	$E_{g,C}$ & $5,47$ & Diamant \\ \hline
	$E_{g,CdS}$ & $2,42$ & Cadmiumsulfid \\ \hline
	$E_{g,GaP}$ & $2,26$ & Galliumphosphid \\ \hline
	$E_{g,GaAs}$ & $1,42$ & Gallium-Arsenid \\ \hline
	$E_{g,InP}$ & $1,35$ & Indiumphosphid \\ \hline
	$E_{g,Si}$ & $1,12$ & Silizium \\ \hline
	$E_{g,Ge}$ & $0,66$ & Germanium \\ \hline
	$E_{g,InSb}$ & $0,17$ & Indiumantimonid \\ \hline
	\end{longtable}
	\renewcommand{\arraystretch}{1}
	
	\subsection{Eckdaten wichtiger Halbleiter} \label{ch:eckd}
	\renewcommand{\arraystretch}{1.5}
	\begin{longtable} {|p{2cm}|p{2.6cm}|p{2.6cm}|p{2.6cm}|p{2.7cm}|} \hline
	% Definition des Tabellenkopfes auf der ersten Seite
	%Spaltenbezeichnungen
	\textbf{Ch. Sym.} & \textbf{$E_g$ in $\bracks{eV}$} & \textbf{$N_C$ in $\bracks{cm^{-3}}$} & \textbf{$N_V$ in $\bracks{cm^{-3}}$} & \textbf{$n_i$ in $\bracks{cm^{-3}}$} \\
	\hline
	\endfirsthead % Erster Kopf zu Ende
	% Definition des Tabellenkopfes auf den folgenden Seiten
	\caption{Eckdaten wichtiger Halbleiter}\\ \hline
	%Spaltenbezeichnungen
	\textbf{Ch. Sym.} & \textbf{$E_g$ in $\bracks{eV}$} & \textbf{$E_g$ in $\bracks{eV}$} & \textbf{$E_g$ in $\bracks{eV}$} & \textbf{$E_g$ in $\bracks{eV}$} \\
	\hline
	\endhead % Zweiter Kopf ist zu Ende
	\multicolumn{3}{r}{Fortsetzung auf Folgeseite}\\
	\endfoot
	\hline
	%\multicolumn{3}{r}{Ende} \\
	\endlastfoot
	Si & $1,124$ & $2,81 \cdot 10^{19}$ & $2,88 \cdot 10^{19}$ & $1,04 \cdot 10^{10}$ \\ \hline
	Ge & $0,67$ & $1,05 \cdot 10^{19}$ & $3,92 \cdot 10^{18}$ & $1,55 \cdot 10^{13}$ \\ \hline
	GaAs & $1,424$ & $4,33 \cdot 10^{17}$ & $8,13 \cdot 10^{18}$ & $2,04 \cdot 10^{6}$ \\ \hline
	\end{longtable}
	\renewcommand{\arraystretch}{1}
	\newpage
\subsection{Niederfeld- und Niederdotierungsbeweglichkeiten ($T = 300K$)} \label{ch:bewegl.}
	\renewcommand{\arraystretch}{1.5}
	\begin{longtable} {|p{2.4cm}|p{3.5cm}|p{3.5cm}|p{3.5cm}|} \hline
	% Definition des Tabellenkopfes auf der ersten Seite
	%Spaltenbezeichnungen
	\textbf{$n/p$} & \textbf{Si} & \textbf{Ge} & \textbf{GaAs} \\
	\hline
	\endfirsthead % Erster Kopf zu Ende
	% Definition des Tabellenkopfes auf den folgenden Seiten
	\caption{Niederfeld- und Niederdotierungsbeweglichkeiten}\\ \hline
	%Spaltenbezeichnungen
	\textbf{$n/p$} & \textbf{Si} & \textbf{Ge} & \textbf{GaAs} \\
	\hline
	\endhead % Zweiter Kopf ist zu Ende
	\multicolumn{3}{r}{Fortsetzung auf Folgeseite}\\
	\endfoot
	\hline
	%\multicolumn{3}{r}{Ende} \\
	\endlastfoot
	  $\mu_n \bracks{\frac{cm^2}{Vs}}$ & $1340$ & $3900$ & $8000$ \\ \hline	
	  $\mu_p \bracks{\frac{cm^2}{Vs}}$ & $460$ & $1900$ & $400$ \\ \hline	
	\end{longtable}
	\renewcommand{\arraystretch}{1}	
	
	\subsection{Konstanten} \label{ch:constants}
	\renewcommand{\arraystretch}{1.5}
	
	\begin{longtable} {|p{0.6cm}|p{4.4cm}|p{8.4cm}|} \hline
	% Definition des Tabellenkopfes auf der ersten Seite
	%Spaltenbezeichnungen
	\textbf{Ze.} & \textbf{Wert} & \textbf{Bedeutung}\\
	\hline
	\endfirsthead % Erster Kopf zu Ende
	% Definition des Tabellenkopfes auf den folgenden Seiten
	\caption{Konstanten}\\ \hline
	%Spaltenbezeichnungen
	\textbf{Ze.} & \textbf{Wert} & \textbf{bedeutung}\\
	\hline
	\endhead % Zweiter Kopf ist zu Ende
	\multicolumn{3}{r}{Fortsetzung auf Folgeseite}\\
	\endfoot
	\hline
	%\multicolumn{3}{r}{Ende} \\
	\endlastfoot
	
	%a-g
	$c$ & $2,998...\cdot 10^8 \bracks{frac{m}{s}}$ & Lichtgeschwindigkeit\\ \hline
	$e,q$ & $1,602176...\cdot 10^{-19}\bracks{C}$ & Elementarladung\\ \hline
	%h-n
	$h$ & $6,63 \cdot 10^{-34} \bracks{Js}$ & Planck-Konstante\\ \hline
	$h$ & $4,136...\cdot 10^{-15} \bracks{eVs}$ & Planck-Konstante\\ \hline
	$\hbar$ & $\frac{h}{2\pi}$ & Plancksches Wirkungsquantum\\ \hline
	$k$ & $8,6173 \cdot 10^{-5} \bracks{\frac{eV}{K}}$ & Boltzmann Konstante\\ \hline
	$kT$ & $25,85 \bracks{meV}$ & mit der Boltzmann Konstante und $T=300K$ \\ \hline
	%m-u
	$m_0$ & $9,11 \cdot 10^{-31} \bracks{kg}$ & Elektronenmasse\\ \hline
	$m^*_{si}$ & $0,2 \cdot m_0$ & Effektive Masse Silizium \\ \hline
	$m^*_{ge}$ & $0,1 \cdot m_0$ & Effektive Masse Germanium \\ \hline
	$N_V$ & $1,04\cdot 10^{19}cm^{-3}$ & Zustandsdichte im VB Silizium \\ \hline
	$N_C$ & $2,80\cdot 10^{19}cm^{-3}$ & Zustandsdichte im LB Silizium \\ \hline
	$R$ & $1,09737 \cdot 10^7 m^{-1} $ & Rydbergkonstante\\ \hline
	%v-z
	
	%griechisch
	$\varepsilon_0$ & $8,854..\cdot 10^{-12}\bracks{\frac{As}{Vm}}$ & Dielektrizitätskonstante des Vakuuums \\ \hline
	$\varepsilon_{Si}$ & $11,90$ & Korrekturfaktor Dielektrizitätskonstante für Silizium\\ \hline
	$\varepsilon_{Ge}$ & $16$ & Korrekturfaktor Dielektrizitätskonstante für Germanium\\ \hline
	$\varepsilon_{Si0_2}$ & $3,9$ & Korrekturfaktor Dielektrizitätskonstante für Si02\\ \hline
	%Sonderzeichen
	\end{longtable}
	\renewcommand{\arraystretch}{1}
  
  \newpage
  \subsection{Nachwort}
  Diese Formelsammlung wurde nahezu ausschließlich auf Basis des Mikroelektronik-I Scripts von Prof. Dr. Jürgen H. Werner und der Mikroelektronik 2 Vorlesung von Prof. Dr. habil. Jörg Schulze erstellt. Nahezu sämtliche Formeln und Werte sind direkt dem Script und der Vorlesung entnommen und wurden nicht für diese Sammlung eigenständig hergeleitet. Für ausführlichere Beschreibungen empfehle ich sehr das eben angesprochene Script zu studieren, dass unter \cite{Mikro1} im Literaturverzeichnis zu finden ist. Es kann direkt im "Kopierlädle" der Universität Stuttgart gedruckt werden. Diese Formelsammlung ist einzige ein Hilfsmittel für mich und meine Kommilitonen und sehr wahrscheinlich nicht fehlerfrei. Sollten Fehler gefunden werden, würde ich mich sehr freuen wenn man mir das kurz in einer E-Mail (f.leuze@outlook.de) mitteilen würde, damit ich entsprechende Korrekturen vornehmen kann. 
  
\bibliography{lit}

\end{document}